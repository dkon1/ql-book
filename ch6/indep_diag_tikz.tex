% Venn diagram with magnifier
% Author: Dennis Heidsiek
\documentclass{minimal}

\usepackage{tikz}
\usetikzlibrary{spy}

\usepackage{verbatim}
\usepackage[active,tightpage]{preview}
\PreviewEnvironment{tikzpicture}

\begin{document}
\pagestyle{empty}

\begin{comment}
:Title: Venn diagram with magnifier

This example shows how to add a glass to magnify a special part of a pictures.
It makes use of the new spy library, so you'll need a recent TikZ version [1]
to compile it.

It was created by Dennis Heidsiek [2], based on the example [3],
inspired by a sketch in [4].

- [1] http://www.texample.net/tikz/builds/
- [2] http://www.google.com/profiles/Dennis.Heidsiek
- [3] http://www.texample.net/tikz/examples/venn-diagram/
- [4] http://wiki.the-big-bang-theory.com/wiki/Psychic_Vortex

\end{comment}

% First, we define three circles:
\def\firstcircle{(-2,0) circle (1.9)}
\def\secondcircle{(2,0) circle (1.9)}
\pgfmathparse{-(2.4^2-2^2)^0.5} % by pythagoras
\let\h\pgfmathresult % shortcut for further use
\def\thirdcircle{(0,\h) circle (0.2cm)}

\colorlet{circle edge}{black!90}
\colorlet{circle area1}{blue!30}
\colorlet{circle area2}{red!30}

\tikzset{filled/.style={fill=circle area1, draw=circle edge, thick},
    outline/.style={draw=circle edge, thick}}


\begin{tikzpicture}
 % Let's draw the scene (to magnify):
  \begin{scope}[spy using outlines=
      {magnification=16, size=8cm, connect spies, rounded corners}]
    
    % the border:
    \draw[thick, rounded corners] (-5,-4) rectangle (5,4);
  %  \draw (0,3.3) node[scale=2] {all people};
    
    % The transparency:
        \begin{scope}
%        \clip \firstcircle;
 	   \begin{scope}[fill opacity=0.5]
      	\fill[red] \firstcircle;
      	\fill[green] \secondcircle;
     % \fill[blue] \thirdcircle;
    \end{scope}

    
    \draw[align=center] \firstcircle node {P=0.2};
    \draw[align=center] \secondcircle node {P=0.2} ;
%\draw (0.4,0) node[align=center] {P=0.1};

    \end{scope}
    \end{scope}
\end{tikzpicture}  
 
 
\def\firstcircle{(-2,0) circle (2.9)}
\def\secondcircle{(2,0) circle (2.5)}
\pgfmathparse{-(2.4^2-2^2)^0.5} % by pythagoras
\let\h\pgfmathresult % shortcut for further use

\begin{tikzpicture}
 % Let's draw the scene (to magnify):
  \begin{scope}[spy using outlines=
      {magnification=16, size=8cm, connect spies, rounded corners}]
    
     the border:
    \draw[thick, rounded corners] (-5,-4) rectangle (5,4);
%    \draw (0,3.3) node[scale=2] {all people};
    
    % The transparency:
        \begin{scope}
%        \clip \firstcircle;
 	   \begin{scope}[fill opacity=0.5]
      	\fill[red] \firstcircle;
      	\fill[green] \secondcircle;
     % \fill[blue] \thirdcircle;
    \end{scope}
    
    \draw[align=center] \firstcircle node {P=0.4};
    \draw[align=center] \secondcircle node {P=0.3} ;
\draw (0.2,0) node[align=center] {P=0.12};

    \end{scope}
    \end{scope}
\end{tikzpicture}   
 
\def\firstcircle{(0,0) circle (3.8)}
\def\secondcircle{(0,0) circle (2)}
\pgfmathparse{-(2.4^2-2^2)^0.5} % by pythagoras
\let\h\pgfmathresult % shortcut for further use 
 
\begin{tikzpicture}
 % Let's draw the scene (to magnify):
  \begin{scope}[spy using outlines=
      {magnification=16, size=8cm, connect spies, rounded corners}]
    
     the border:
    \draw[thick, rounded corners] (-5,-4) rectangle (5,4);
%    \draw (0,3.3) node[scale=2] {all people};
    
    % The transparency:
        \begin{scope}
%        \clip \firstcircle;
 	   \begin{scope}[fill opacity=0.5]
      	\fill[red] \firstcircle;
      	\fill[green] \secondcircle;
     % \fill[blue] \thirdcircle;
    \end{scope}
    
    \draw (0,3) node {P=0.7};
    \draw[align=center] \secondcircle node {P=0.3} ;
%\draw (0.2,0) node[align=center] {P=0.12};

    \end{scope}
    \end{scope}
\end{tikzpicture}   



\end{document}
